\documentclass[lang=cn,11pt,a4paper,cite=authoryear]{elegantpaper}

% 微分号
\newcommand{\dd}[1]{\mathrm{d}#1}
\newcommand{\pp}[1]{\partial{}#1}

\newcommand{\homep}[1]{\section*{Problem #1}}

% FT LT ZT
\newcommand{\ft}[1]{\mathscr{F}[#1]}
\newcommand{\fta}{\xrightarrow{\mathscr{F}}}
\newcommand{\lt}[1]{\mathscr{L}[#1]}
\newcommand{\lta}{\xrightarrow{\mathscr{L}}}
\newcommand{\zt}[1]{\mathscr{Z}[#1]}
\newcommand{\zta}{\xrightarrow{\mathscr{Z}}}

% 积分求和号

\newcommand{\dsum}{\displaystyle\sum}
\newcommand{\aint}{\int_{-\infty}^{+\infty}}

% 简易图片插入
\newcommand{\qfig}[3][nolabel]{
  \begin{figure}[!htb]
      \centering
      \includegraphics[width=0.6\textwidth]{#2}
      \caption{#3}
      \label{#1}
  \end{figure}
}

% 表格
\renewcommand\arraystretch{1.5}


% 日期


\title{微电子器件实验\quad 基本运算电路}
\author{范云潜, 学号:18373486,搭档:徐靖涵,教师:彭守仲}
\institute{微电子学院 184111 班}
\date{\zhtoday}

\begin{document}

\maketitle

% \tableofcontents

\section{实验目的}

在之前课程的基础上,本组实验通过对运放的负反馈设计完成对信号的基本算数处理,如差分放大、微分、指数、对数等,加深对运放使用与信号负反馈的理解,掌握设计与使用能力。

\section{实验所用设备及器件}


主要设备有:电压源,任意波形发生器,示波器,台式万用表,相关线缆等,主要器件有四运放集成电路LM324N、电容、电阻,限流二极管。

\section{实验基本原理及步骤}

\subsection{差分放大电路}

差分放大电路是一种特殊的加减运算电路,因此可以利用上一周实验的理论结果进行搭建,只需将两条支路的输入电阻分别设置为反馈电阻的十分之一,并且保持运放两个输入端的电阻匹配即可。这里取 1k 和 10k ,如 \figref{01} 。

\qfig[01]{1501.png}{差分运算电路}

\subsection{微分运算电路}

类似的,为了拼凑微分电路,需要将输入进行微分,而非输出,只需将积分电路的负端与输出端的电容、电阻交换位置即可:

\[v_o =  - R_f C \frac{\dd{v_i}}{\dd{t}}\]

选择 \(R = 100k \Omega, C = 10 \mu F\) ,如 \figref{02} 。

\qfig[02]{1502.png}{微分运算电路}

\subsection{指数运算电路}

与微分-积分关系类似的,可以设计指数-对数电路,只需实现电流为输入的指数即可得到指数运算电路,为此,将二极管作为负载放在负端输入,同时,考虑到电路的输出电压受限,同等电流情况下\footnote{对于二极管来说,其电路只和输入电压有关,因为运放的输入端已经虚地。},负载电阻应当尽量小,选择 \(100 \Omega\) ,如\figref{03}。

\qfig[03]{1503.png}{指数运算电路}

此时的电学关系满足:

\[v_o = - R I_s \exp \frac{v_i}{u_t} \]

\subsection{对数运算电路}

基于以上的分析,同等输入电压下,为了获得较大的输出电压,需要通过二极管电流较大,因此也需要较小的输入电阻,选择 \(100 \Omega\) ,如\figref{04} 。

\qfig[04]{1504.png}{对数运算电路}

此时满足的电学关系为:

\[I_s \exp \frac{0 - v_o}{u_t} = \frac{v_i}{R} \]

整理得到:

\[v_o = - u_t \ln \frac{v_i}{I_s R}\]

\subsection{实验步骤}

根据以上分析,搭建相对应的电路,将运放的供电分别设置为 \(\pm 5 V\) 。

设置好外围电路后,对于差分放大电路以及微分运算电路,使用任意波形发生器,观测、记录响应的输出波形;对于指数、对数运算电路,使用电压源作为输入,测量记录输出后进行图像的拟合。

\section{实验数据记录}

原始数据请见 \href{https://github.com/PannenetsF/Mirco-Electronic-Device-Experiment/tree/main/homework/hw15}{这里} 。

需要注意的是,本小节的拟合图像均将输入放大了 \(1e6\) 倍。

\subsection{指数运算电路}

将其输入与输出进行了指数拟合,如\figref{05}。对比上一小节推导的公式,得到 \(I_s = 13.9 nA\) , \(u_t = 53.856 mV\) 。

\qfig[05]{1505.png}{指数运算电路拟合结果}


\subsection{对数运算电路}

将其输入与输出进行了对数拟合,如\figref{06}。对比上一小节推导的公式,得到 \(u_t = 57.331 mV\) , \(I_s = 30 nA\) 。

\qfig[06]{1506.png}{对数运算电路拟合结果}

\section{实验结果分析}

基本来说,本次实验电路结果较为稳定,并且可以复现,验证了已有知识。

\section{总结与思考}


Q1. 任意波形发生器的内阻与微分电路中的电容组成一个什么
样的滤波器?截止频率是多大?

其电路如 \figref{07} 。满足:

\[A_v = \frac{1/sC}{1/sC + R} = \frac{1}{1 + s C R}\] 

那么:

\[|A_v| = \frac{1}{\sqrt{1 + \omega^2 C^2 R^2}} = \frac{1}{\sqrt{2}}\]

解得:

\[\omega = \frac{1}{RC} = \frac{1}{500 \mu} = 2000\] 

约为 \(320 Hz\) 。

\qfig[07]{1507.png}{滤波器结构}

Q2. 方波信号经过低通滤波后得到什么样的波形?为什么?

应该是类似 \(\mathit{sa}\) 函数构成的梯形的形状。方波的频域表示为 \(\mathit{sa}\) 函数,低通滤波器(类似方波)将其主要幅度特性留下,恢复出一个缺少高频低幅度的“方波”。

Q3. 输入电压在什么范围内时指数运算和对数运算电路能够正
常工作?



% Start Here

% End Here

\end{document}