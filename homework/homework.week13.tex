\documentclass[lang=cn,11pt,a4paper,cite=authoryear]{elegantpaper}

% 微分号
\newcommand{\dd}[1]{\mathrm{d}#1}
\newcommand{\pp}[1]{\partial{}#1}

\newcommand{\homep}[1]{\section*{Problem #1}}

% FT LT ZT
\newcommand{\ft}[1]{\mathscr{F}[#1]}
\newcommand{\fta}{\xrightarrow{\mathscr{F}}}
\newcommand{\lt}[1]{\mathscr{L}[#1]}
\newcommand{\lta}{\xrightarrow{\mathscr{L}}}
\newcommand{\zt}[1]{\mathscr{Z}[#1]}
\newcommand{\zta}{\xrightarrow{\mathscr{Z}}}

% 积分求和号

\newcommand{\dsum}{\displaystyle\sum}
\newcommand{\aint}{\int_{-\infty}^{+\infty}}

% 简易图片插入
\newcommand{\qfig}[3][nolabel]{
  \begin{figure}[!htb]
      \centering
      \includegraphics[width=0.6\textwidth]{#2}
      \caption{#3}
      \label{#1}
  \end{figure}
}

% 表格
\renewcommand\arraystretch{1.5}


% 日期


\title{微电子器件实验\quad 源随器与共栅放大器参数提取}
\author{范云潜, 学号:18373486,搭档:徐靖涵,教师:彭守仲}
\institute{微电子学院 184111 班}
\date{\zhtoday}

\begin{document}

\maketitle

% \tableofcontents

\section{实验目的}


通过对不同连接方式的场效应晶体管放大器的参数进行理论分析与实际测量,加深对不同放大器特点与功能的认识,掌握对小信号电路的分析能力。


\section{实验所用设备及器件}

主要设备有:电压源,任意波形发生器,示波器,手持式万用表,台式万用表,相关线缆等,主要器件有 晶体管 IRFU214 与电阻。

\section{实验基本原理及步骤}

\subsection{源随器电路推导}

电路小信号如 \figref{01} 。

\qfig[01]{1301.png}{源随器小信号电路}

接下来进行推导: 

\[\begin{aligned}
    g_m(v_{in} - v_{out}) R_S &= v_{out} \\
    A_v &= \frac{v_{out}}{v_{in}} = \frac{g_m R_S}{1 + g_m R_S} \\ 
    R_{in} = \frac{i_{in}}{v_{in}} &= R_G \\ 
    R_{out} = \frac{v_{out}}{i_{out}} &= \frac{1}{g_m + 1/R_S} \\ 
    A_i = \frac{i_{out}}{i_{in}} =  \dfrac{\dfrac{R_L}{R_S + R_L} g_m (v_{in} - v_{out})}{v_{in} / R_G} &= \dfrac{R_G g_m \dfrac{R_S}{R_S + R_L}}{ (1 + g_m {R_S // R_L})}
\end{aligned}\]

\subsection{实验步骤}

首先测量源随器的直流特性: 

\begin{enumerate}
    \item 按照 \figref{02} 搭建电路
    \item 调节 \(E_G = 5 V\) 
    \item 调节 \(E_D = 0.1 - 8 V\) 
    \item 测量 \(V_{DS}\) 和 \(I_D\) 并绘制图像
\end{enumerate}

\qfig[02]{1302.png}{源随器直流工作点小信号电路}

根据图像,得到放大电路合适的直流工作点,并进行放大电路的测量:

\begin{enumerate}
    \item 按照 \figref{03} 搭建电路
    \item 按照上一步设好的直流工作点设置 \(E_D\) 和 \(E_G\) 
    \item 任意波形发生器输出 \(1 kHz 500 m Vpp\)  的正弦信号 \(v_{in}\) 
    \item 断开 \(R_L\) ,用示波器测量 \(v_{in}\) 和 \(v_{out,1}\) ,用万用表测量 \(i_{in}\) 和 \(i_d\) 
    \item 连接 \(R_L\) 测量 \(v_{out,2}\)
\end{enumerate}


\qfig[03]{1303.png}{源随器放大电路}

\section{实验数据记录}



原始数据请见 \href{https://github.com/PannenetsF/Mirco-Electronic-Device-Experiment/tree/main/homework/hw13}{这里} 。

\subsection{实验一}

\qfig[04]{1304.png}{源随器直流特性 \(V_{DS}-I_D\)} 

直流特性如 \figref{04} 所示,据此设置 \(E_G = E_D = 5 V\) ,得到的交流特性 \tabref{t1} 。

\begin{table}[htb]
    \centering
    \caption{源随器特性测量}
    \label{t1}
    \begin{tabular}{|c|c|c|c|c|c|c|c|c|c|}
    \hline
    EG & ED & vin      & vout     & iin      & id       & noload   & AV          & RIN         & gm          \\ \hline
    5  & 5  & 1.78E-01 & 1.63E-01 & 1.50E-06 & 1.64E-04 &          & 0.915 & 118666 & 0.0109 \\ \hline
       &    &          &          &          &          & withload & AI          & ROUT        &             \\ \hline
       &    & 1.78E-01 & 1.51E-01 & 1.50E-06 & 3.03E-04 &          & 100.6 & 79.47 &             \\ \hline
    \end{tabular}
\end{table}

\section{实验结果分析}

源随器的电压增益接近 \(1\), 输入电阻较大,而电流增益较大,输出电阻小,因此可以很好接受电压并传递给下一级,并放大电流。

\section{总结与思考}

Q1:根据之前的理论结果, \(A_v = 0.9167\) , \(A_i = 84.615\) , \(R_{in} = 100 k \Omega\) , \(R_{out} = 83.3 \Omega\) ,和测量值较为接近。

% Start Here

% End Here

\end{document}