\documentclass[lang=cn,11pt,a4paper,cite=authoryear]{elegantpaper}

% 微分号
\newcommand{\dd}[1]{\mathrm{d}#1}
\newcommand{\pp}[1]{\partial{}#1}

\newcommand{\homep}[1]{\section*{Problem #1}}

% FT LT ZT
\newcommand{\ft}[1]{\mathscr{F}[#1]}
\newcommand{\fta}{\xrightarrow{\mathscr{F}}}
\newcommand{\lt}[1]{\mathscr{L}[#1]}
\newcommand{\lta}{\xrightarrow{\mathscr{L}}}
\newcommand{\zt}[1]{\mathscr{Z}[#1]}
\newcommand{\zta}{\xrightarrow{\mathscr{Z}}}

% 积分求和号

\newcommand{\dsum}{\displaystyle\sum}
\newcommand{\aint}{\int_{-\infty}^{+\infty}}

% 简易图片插入
\newcommand{\qfig}[3][nolabel]{
  \begin{figure}[!htb]
      \centering
      \includegraphics[width=0.6\textwidth]{#2}
      \caption{#3}
      \label{#1}
  \end{figure}
}

% 表格
\renewcommand\arraystretch{1.5}


% 日期


\title{微电子器件实验\quad 双极型晶体管的直流特性测量与分析}
\author{范云潜, 学号:18373486,搭档:徐靖涵,教师:彭守仲}
\institute{微电子学院 184111 班}
\date{\zhdate{2020/10/19}}

\begin{document}

\maketitle

% \tableofcontents

\section{实验目的}

双极型晶体管的直流特性测量与分析,并从中:

\begin{itemize}
    \item 了解通用仪表的的基本原理和使用方法
    \item 了解被测器件各项参数的定义和测量方法
    \item 掌握被测器件直流特性和相关机理
\end{itemize}

\section{实验所用设备及器件}

实验用到的核心器件是双极型晶体管 9018 ,其他设备包括面包板、杜邦线、电压源、手持万用表、台式万用表等。

\section{实验基本原理及步骤}

基本原理是三极管的两个 PN 结的特性,因此基极电流基本和二极管一致,而 \(V_{CE}\) 的变化可以使得同等的基极电压下电流增大。

\begin{enumerate}
    \item 搭建电路,电路图如\figref{01}
    \item 设 \(V_{CE} = 0\) ,调节 \(V_B\) ,记录电压 \(V_{BE}\) 以及电流 \(I_B\) 。
    \item 设 \(V_{CE} = 0.5 V\) ,调节 \(V_B\) ,记录电压 \(V_{BE}\) 以及电流 \(I_B\) 。
    \item 拆解电路,还原仪器
\end{enumerate}

\qfig[01]{0702p1.png}{三极管测试电路}

\section{实验数据记录}

原始数据可视化后如\figref{02},\figref{03} ,原始数据请见 \href{https://github.com/PannenetsF/Mirco-Electronic-Device-Experiment/tree/main/homework/hw07/01led}{这里} 。

\qfig[02]{0702p2.png}{\(V_{CE} = 0 V\) 时 \(I-V_{B}\) 曲线}

\qfig[03]{0702p3.png}{\(V_{CE} = 0.5 V\) 时 \(I-V_{B}\) 曲线}

\section{实验结果分析}

三极管中的 PN 结表现出类似二极管的伏安特性,和理论相符:在正向导通时,电流随电压成指数变化,较小电压变化会引起较大的电流变化;反向导通且未击穿时,较大的电压变化引起的电流增大较小。在集电极偏置变大后,漂移电流增大,导致发射极的电流得到进一步放大。

\section{总结与思考}

Q:VCE电压是如何影响BE端的伏安特性曲线的?内在机理是什么?

放大的原理:
发射区重掺杂、发射结正偏,大量电子从发射区扩散到基区。
基区很薄、多子浓度低,所以极少数从发射区扩散来的电子与基区的空穴复合。
集电区面积大、基区很薄、集电结反偏,那么大部分从发射区扩散来的电子能漂移到集电区,阻碍很小。

在这种情况下, \(V_{CE}\) 的增大会增大集电区的漂移,进而使得电流增大。
% Start Here

% End Here

\end{document}