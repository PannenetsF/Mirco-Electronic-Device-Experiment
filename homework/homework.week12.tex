\documentclass[lang=cn,11pt,a4paper,cite=authoryear]{elegantpaper}

\input{needed.tex}

\title{微电子器件实验\quad 共集极与共基极电路参数测量}
\author{范云潜, 学号:18373486,搭档:徐靖涵,教师:彭守仲}
\institute{微电子学院 184111 班}
\date{\zhtoday}

\begin{document}

\maketitle

\section{实验目的}

通过对不同连接方式的双极型晶体管放大器的参数进行理论分析与实际测量,加深对不同放大器特点与功能的认识,掌握对小信号电路的分析能力。

\section{实验所用设备及器件}

主要设备有:电压源,任意波形发生器,示波器,手持式万用表,台式万用表,相关线缆等,主要器件有 晶体管 C9018 与电阻。

\section{实验基本原理及步骤}

\subsection{共集极电路推导}

电路小信号如 \figref{01} 。

\[\begin{aligned}
    R_{in} &= \frac{v_{i}}{i_{i}} \\ 
    R_{out} &= \frac{v_o' - v_o}{v_o / R_L} \\ 
    A_v &= \frac{u_o}{u_i}, \text{ no load} \\
    A_i &= \frac{i_o}{i_i} = \frac{- i_L - i_e + (1+\beta) i_b}{i_i}, i_b = i_{i} - v_i / R_b, \text{ with load}
\end{aligned}\] 

\qfig[01]{1201.png}{共集极小信号电路}

\subsection{共基极电路推导}


电路小信号如 \figref{02} 。

\[\begin{aligned}
    R_{in} &= \frac{v_{i}}{i_{i}} \\ 
    R_{out} &= \frac{v_o' - v_o}{v_o / R_L} \\ 
    A_v &= \frac{u_o}{u_i}, \text{ no load} \\
    A_i &= \frac{i_o}{i_i} = \frac{u_o / R_L}{i_i} , \text{ with load}
\end{aligned}\] 


\qfig[02]{1202.png}{共基极小信号电路}


\subsection{实验步骤}

整体流程都是确保晶体管工作在放大区后,在进行小信号的测量。

首先是共集极电路的直流特性,电路如 \figref{03} ,步骤为:

\begin{enumerate}
    \item 调节 \(E_B\) ,使得 \(I_B = 60 \mu A\) 
    \item 调节 \(E_C\) ,在 \(0.1 - 1 V\) 与 \(1 - 15 V\) 分别均匀取点
    \item 测量对应的 \(V_{CE}\) 、 \(I_C\) 并画图,计算 \(I_C / I_B\) 
\end{enumerate}

\qfig[03]{1203.png}{共集极直流电路}

放大电路参数测量,修改电路如 \figref{04}:

\begin{enumerate}
    \item \(E_B\) 保持不变, \(E_C = 12 V \) 
    \item 任意波形发生器输出 \(1 kHz, 1 V_{pp }\)  的电压
    \item 示波器测量 \(v_{i}\) 与 \(R_L\) 上的电压波形,万用表测量 \(i_i\) 
    \item 断开、连接 \(R_L\) 分别测量一组数据
\end{enumerate}

\qfig[04]{1204.png}{共集极交流电路}

接下来是共基极电路的直流特性,电路如 \figref{05} ,步骤为:

\begin{enumerate}
    \item 调节 \(E_B\) ,使得 \(I_B = 60 \mu A\) 
    \item 调节 \(E_C\) ,在 \(0.1 - 1 V\) 与 \(1 - 15 V\) 分别均匀取点
    \item 测量对应的 \(V_{CE}\) 、 \(I_C\) 并画图,计算 \(I_C / I_B\) 
\end{enumerate}


\qfig[05]{1205.png}{共基极直流电路}

放大电路参数测量,修改电路如 \figref{06}:

\begin{enumerate}
    \item \(E_B\) 保持不变, \(E_C = 12 V \) 
    \item 任意波形发生器输出 \(1 kHz, 1 V_{pp }\)  的电压
    \item 示波器测量 \(v_{i}\) 与 \(R_L\) 上的电压波形,万用表测量 \(i_i\) 
    \item 断开、连接 \(R_L\) 分别测量一组数据
\end{enumerate}

\qfig[06]{1206.png}{共基极直流电路}


\section{实验数据记录}

原始数据请见 \href{https://github.com/PannenetsF/Mirco-Electronic-Device-Experiment/tree/main/homework/hw12}{这里} 。

\subsection{实验一}

共集极电路的特性曲线如 \figref{07} ,最终 \(I_C/I_B\) 稳定在 \(104\) 。交流特性测量如 \tabref{t1} 。

\qfig[07]{1207.png}{共集极电路的 \(V_{CE}-I_C\) 变化图} 

\begin{table}[htb]
\centering
\caption{共集极特性测量}
\label{t1}
\begin{tabular}{|l|l|l|l|l|l|l|l|}
\hline
EB  & EC & IB       & IC       & vb       & vrl      & iin      & type \\ \hline
6.8 & 12 & 3.04E-05 & 3.03E-03 & 3.60E-01 & 3.50E-01 & 6.70E-06 & rl   \\ \hline
6.8 & 12 & 3.04E-05 & 3.03E-03 & 3.60E-01 & 3.55E-01 & 1.01E-05 & no   \\ \hline
\end{tabular}
\end{table}

\subsection{实验二}

共集极电路确认工作在放大区后,交流特性测量如 \tabref{t2} 。

\begin{table}[htb]
\centering
\caption{共基极特性测量}
\label{t2}
\begin{tabular}{|l|l|l|l|}
\hline
noload & vout     & vin      & iin      \\ \hline
       & 9.10E-01 & 8.08E-03 & 9.22E-04 \\ \hline
load   & vout     & vin      & iin      \\ \hline
       & 4.61E-01 & 1.05E-02 & 9.24E-04 \\ \hline
\end{tabular}
\end{table}

\section{实验结果分析}

\subsection{实验一}

根据前面分析的内容,进行计算: 

\[\begin{aligned}
    R_{in} &=   35643 \Omega\\
    R_{out} &= \ = 14.286 \Omega  \\
    A_v &= 0.986 \\
    A_i &= 52.238 
\end{aligned}\]

可以看到,输入电阻较大,输出电阻较小,几乎不放大电压,而是放大电流。

\subsection{实验二}


根据前面分析的内容,进行计算: 

\[\begin{aligned}
    R_{in} &=   8.76 \Omega\\
    R_{out} &= 973.96 \Omega  \\
    A_v &= 112.6 \\
    A_i &= 0.498 
\end{aligned}\]

可以看到,输入电阻较小,输出电阻较大,几乎不放大电流,而是放大电压。

\section{总结与思考}

Q1: 

在共集极电路中,如 \figref{01} , 

\[\begin{aligned}
    R_{in} &=  \frac{r_\pi + (1+\beta)R_e}{1 + (r_\pi + (\beta + 1) R_e) / R_b}  = \frac{870 + 105 \times 1000}{1 + (870 + 105 \times 1000) / 1000} = 51425.65697 \Omega\\
    R_{out} &= R_e // r_\pi // (r_\pi / \beta) = 7.8 \Omega\\ 
    A_v &= \frac{(1+\beta) R_e}{r_\pi + (\beta + 1) R_e} = 0.99178 \\
    A_i &= \frac{1}{2}\frac{(1+\beta) R_b}{1 + (r_\pi + (1+\beta) R_e)} = 49.589
\end{aligned}\]


在共基极电路中,如 \figref{02} , 

\[\begin{aligned}
    R_{in} &= R_e // r_\pi // (r_\pi / \beta) = 7.8 \Omega\\
    R_{out} &= R_e = 1000 \Omega \\ 
    A_v &= \frac{\beta R_c}{r_\pi}= 120.69 \\ 
    A_i &= \frac{1}{2}\frac{\beta}{\beta + 1 + r_\pi / R_e}  = 0.49125
\end{aligned}\]

Q2: 共集极电路输入电阻很大,可以接受前一级的大部分电压,输出电阻很小,可以给出大部分输出电压,几乎不放大电压但是放大电流,作为缓冲器增强驱动能力。共基极电路输入电阻很小,接受大部分电流,但是输出电阻很大,输出大部分电压,有电压放大作用,几乎不放大电流,可以做放大器。

% Start Here

% End Here

\end{document}