\documentclass[lang=cn,11pt,a4paper,cite=authoryear]{elegantpaper}

% 微分号
\newcommand{\dd}[1]{\mathrm{d}#1}
\newcommand{\pp}[1]{\partial{}#1}

\newcommand{\homep}[1]{\section*{Problem #1}}

% FT LT ZT
\newcommand{\ft}[1]{\mathscr{F}[#1]}
\newcommand{\fta}{\xrightarrow{\mathscr{F}}}
\newcommand{\lt}[1]{\mathscr{L}[#1]}
\newcommand{\lta}{\xrightarrow{\mathscr{L}}}
\newcommand{\zt}[1]{\mathscr{Z}[#1]}
\newcommand{\zta}{\xrightarrow{\mathscr{Z}}}

% 积分求和号

\newcommand{\dsum}{\displaystyle\sum}
\newcommand{\aint}{\int_{-\infty}^{+\infty}}

% 简易图片插入
\newcommand{\qfig}[3][nolabel]{
  \begin{figure}[!htb]
      \centering
      \includegraphics[width=0.6\textwidth]{#2}
      \caption{#3}
      \label{#1}
  \end{figure}
}

% 表格
\renewcommand\arraystretch{1.5}


% 日期


\title{微电子器件实验\quad 电压比较器与波形处理电路}
\author{范云潜, 学号:18373486,搭档:徐靖涵,教师:彭守仲}
\institute{微电子学院 184111 班}
\date{\zhtoday}

\begin{document}

\maketitle

% \tableofcontents

\section{实验目的}

之前的系列实验基于运算放大器的线性区(放大区)进行了运算电路的相关特性测量与研究,本实验对运算放大器的非线性去(饱和区)进行进一步的测量与研究,借此加深对运算放大器原理及其功用的理解。

\section{实验所用设备及器件}

主要设备有:电压源,任意波形发生器,示波器,台式万用表,相关线缆等,主要器件有四运放集成电路LM324N、电容、电阻,稳压二极管。

\section{实验基本原理及步骤}

\subsection{电压比较器}

在运算电路的设计中,对于运放的正负输入始终有 \textbf{虚断} 、 \textbf{虚短} 两个假设,从而可以顺利完成从两端较小差分电压到输出端较大变化: \(V_{out} = A_v V_{in}\) 。分析其电路发现其电路均存在反馈回路,电流、电压不会直接作用到输入端,而是通过电阻或者二极管进行分流、分压,从而可以满足假设。

若是将反馈电阻去除,那么电路的假设就不再成立,此时运放两端的差分输入不再是接近 \(0\) 的小量,根据运放的放大倍数,输出会超出运放的供电电压,因此会被钳制在供电电压,失去放大作用,进入饱和区。

接下来进行相关电路转换点的理论推导:

过零比较器的电路结构如 \figref{01} ,没有任何的电阻掺入。

\[
\begin{aligned}
    u_i \text{ 从低转向高:} & u_{-} = u_i = u_+ = 0 = U_{T+} \\ 
    u_i \text{ 从高转向低:} & u_{-} = u_i = u_+ = 0 = U_{T-} \\ 
\end{aligned}    
\]

\qfig[01]{1601.png}{过零比较器电路结构}


单限比较器的电路结构如 \figref{02} ,没有任何的电阻进行电压钳制。可以看出,其参考的运放正极电压保持恒定,这导致其转换阈值只有一个。

\[
\begin{aligned}
    u_i \text{ 从低转向高:} & u_{-} = \dfrac{U_{ref}}{R_1 + R_2} R_2 + \dfrac{u_{i}}{R_1 + R_2} R_1 = u_{+} = 0\\  
    & u_{i+} = U_{T+} = -\dfrac{R_2}{R_1} U_{ref}\\ 
    u_i \text{ 从高转向低:} & u_{+} \dfrac{U_{ref}}{R_1 + R_2} R_2 + \dfrac{u_{i}}{R_1 + R_2} R_1 = u_{+} = 0\\ 
    & u_{i-} = U_{T-} = -\dfrac{R_2}{R_1} U_{ref}\\ 
\end{aligned}    
\]

\qfig[02]{1602.png}{单限比较器电路结构}

滞回比较器的电路结构如 \figref{03} ,其参考电压是随着运放输出而变化的,其参考电压的不一致导致了滞回特性。易知,在输入为低时,输出应为高,反之同理\footnote{这里需要假定输入低到可以比正极的参考电压低,或者高到比其高。}。


\[
\begin{aligned}
    u_i \text{ 从低转向高:} & u_{-} = u_{i+} = u_{+} = \dfrac{R_1}{R_1 + R_2} U_{OH}\\  
    & u_{i+} = U_{T+} = \dfrac{R_1}{R_1 + R_2} U_{OH} \\ 
    u_i \text{ 从高转向低:} &  u_{-} = u_{i-} = u_{+} = \dfrac{R_1}{R_1 + R_2} U_{OL}\\ 
    & u_{i-} = U_{T-} = \dfrac{R_1}{R_1 + R_2} U_{OL} \\ 
    & \text{ 其中,} U_{OL} \text{ 和 } U_{OH} \text{为系统的输出电压,与电源电压并不一致。}
\end{aligned}    
\]

\qfig[03]{1603.png}{滞回比较器电路结构}

将 \figref{03} 中的接地转换为 \(U_{ref}\) 即可得到带参考电压的滞回比较器,类似的:

\[
\begin{aligned}
    u_i \text{ 从低转向高:} & u_{-} = u_{i+} = u_{+} = \dfrac{R_1}{R_1 + R_2} U_{OH} + \dfrac{R_2}{R_1 + R_2} U_{ref}\\  
    & u_{i+} = U_{T+} = \dfrac{R_1}{R_1 + R_2} U_{OH}  + \dfrac{R_2}{R_1 + R_2} U_{ref}\\ 
    u_i \text{ 从高转向低:} &  u_{-} = u_{i-} = u_{+} = \dfrac{R_1}{R_1 + R_2} U_{OL} + \dfrac{R_2}{R_1 + R_2} U_{ref}\\ 
    & u_{i-} = U_{T-} = \dfrac{R_1}{R_1 + R_2} U_{OL}  + \dfrac{R_2}{R_1 + R_2} U_{ref}\\ 
    & \text{ 其中,} U_{OL} \text{ 和 } U_{OH} \text{为系统的输出电压,与电源电压并不一致。}
\end{aligned}    
\]


\subsection{波形发生器}

典型的用于生成波形的结构或器件有 RC 振荡电路、 LC 振荡电路与石英晶振等。本次实验使用到的 RC 振荡电路基于正反馈。

\subsubsection{正反馈系统}

对于如 \figref{04} 的带反馈的系统,通常存在负反馈、正反馈两种形式,其中负反馈是为了增强系统对于干扰的抵抗能力,而正反馈则是使得系统迅速达到某一状态的手段。

对于此系统,其稳态为:

\[A(s) V_f \beta(s) = V_f\]

因此需要在变换域中需要满足 \(A(s) \beta(s) = 1\) ,显然此关系不会普遍成立,因此最后会留存 \(A(s_0) \beta(s_0) = 1\)  的分量,其变换域频率单一,在时域表现即为单一频率的正弦波。当然,为了开始振荡,需要 \(A(s) \beta(s) > 1\) 。

对于存在稳态的系统来说,正反馈可以使得多种稳态的切换更加迅速,从而得到更加理想的系统输出,或者使单稳态系统的瞬态不稳定存续时间更短等。

\qfig[04]{1604.png}{反馈系统基本结构}

\subsubsection{RC 振荡电路}

根据上一小节的分析, 对于本次实验的电路,如 \figref{05} ,输入可视为 \(\dot{U_p}\),其开环增益为:

\[A(s) = \dfrac{\dot{U_O}}{\dot{U_P}} = \dfrac{R_1 + R_F}{R_1}\]

其反馈系数为: 

\[\begin{aligned}
    \beta(s) &= \dfrac{R // \dfrac{1}{sC}}{R + {1}/{sC} + R // {1}/{sC}} \\ 
    &= \frac{R / sC}{(R + 1 / sC)^2 + R / sC} \\ 
    &= \frac{1}{3 + RsC + 1 / RsC}
\end{aligned} \]

那么其幅值为:

\[|\beta| = \frac{1}{\sqrt{9 + (RC\omega - 1/RC\omega)^2}}\]

为了保持其幅值与相位,振荡频率为本征频率 \(\omega = 1 / RC\) ,此时幅度为 \(1 / 3\) 。为了保持其环路增益大于 \(1\),那么 \(\dfrac{R_1 + R_F}{R_1} > 3\) ,因此 \(R_F > 2 R_1\) 。

\qfig[05]{1605.png}{RC 桥式正弦波振荡电路}

\subsubsection{滞回比较器}

为了将正弦波整流成方波,需要具有滞回特性的电路,选用滞回比较器,如 \figref{06} 。其中 \(R = 50 \Omega\) , \(R_1 = R_2 = 1 k\Omega\) 。

为了使得稳压二极管工作在稳压状态,需要通过电路在约 \(50 mA\) 的位置。假设此时输出电压(已经稳压)为 \(u_o = 7.5 V\) ,运放输出端为 \(10 V\) ,那么:

\[\frac{10 - 7.5}{R} = \frac{7.5}{2 k} + 50 m\]

可以解得 \(R = 46.512 \Omega\) ,那么取 \(50 \Omega\) 是合理的。

\qfig[06]{1606.png}{滞回比较器}

\subsubsection{积分电路} 

基本原理在前序实验中有详细介绍,但是需要注意,若是 \(R C\) 常数越大,积分数值就会越小,对应的噪声越大。

\section{实验数据记录}

\subsection{过零比较器}

如 \figref{07} 。

\qfig[07]{1607.jpg}{过零比较器波形}

\subsection{单限比较器}

\qfig[08]{1608.jpg}{单限比较器波形\(U_T = -2 V\)}

\subsection{滞回比较器}

如 \figref{09} 。

\qfig[09]{1609.jpg}{滞回比较器波形 \(\Delta U = 5 V\)}

\subsection{带参考电压的滞回比较器}


如 \figref{10} 。

\qfig[10]{1610.jpg}{带参考电压的滞回比较器波形\(U_{T1} = -1.5 V, U_{T2} = 3.5 V\)}

\subsection{方波转换}

如 \figref{11} 。

\qfig[11]{1611.jpg}{正弦波-方波转换}

\subsection{三角波转换}

如 \figref{12} 。

\qfig[12]{1612.jpg}{方波-三角波转换}

\section{实验结果分析}

以上图形表征着电路基本符合预期设计,接下来对实验中遇到的相关问题进行分析。

首先,由于相关比较器设计中的阈值电压计算时的参数涉及到 \(U_{OH}\) 和 \(U_{OL}\) ,而这两个电压在实际中均无法达到电源电压,因此需要在操作时调节电源电压使输出电压达到设计的预期。

另外,在波形发生器设计中,由于积分电路在电压过高会使得运放进入饱和区,因此无法表现出三角波特性,为此,需要将电源电压提高,运放的工作电压是 \(33 V\) ,取 \(30 V\) 是合适的。


\section{总结与思考}


Q1. 回顾前几节课学到的内容,通过电压比较器得到方波后,
如何得到三角波?

对方波进行积分可以得到三角波。

Q2. 设计一个简单的报警器,当输入电压超过3V时将LED灯
点亮进行报警。

利用单限比较器实现即可,在输入大于 \(3 V\) 时输出为高。
% Start Here

% End Here

\end{document}