\documentclass[lang=cn,11pt,a4paper,cite=authoryear]{elegantpaper}

\input{needed.tex}

\title{微电子器件实验\quad 场效应晶体管交直流特性测量与分析}
\author{范云潜, 学号:18373486,搭档:徐靖涵,教师:彭守仲}
\institute{微电子学院 184111 班}
\date{\zhtoday}

\begin{document}

\maketitle

% \tableofcontents

\section{实验目的}

在之前的实验中,我们对场效应管的直流特性如伏安特性、转移特性进行了测量,作为放大器的基本元件,其放大作用起源于跨导,这标志着一个器件将电压转换成电流的能力。本次实验将从大信号模型的微分以及小信号模型比值两个角度进行测量。

\section{实验所用设备及器件}


主要设备有:电压源,任意波形发生器,示波器,手持式万用表,台式万用表,相关线缆等,主要器件有 N 沟道 MOS 管 IRFR214 。

\section{实验基本原理及步骤}

\subsection{基本原理}

一个 MOS 管可以当作一个简单的放大器,输入是栅源电压 \(V_{GS}\) ,输出是漏极电流 \(I_{D}\) ,而放大的系数正是 MOS 管的跨导。

根据其定义,跨导可以表示为 : 

\[g_m = \frac{\Delta I_{out}}{\Delta V_{in}} = \frac{\dd{I_{out}}}{V_{in}} \]  

因此可以通过直流特性的导函数以及交流信号的比值求出跨导。

另一方面,随频率的上升,馈通效应会逐渐增强,放大器的增益会逐渐减小,降低到原有增益的 \(\dfrac{1}{\sqrt{2}}\) 时,称其到达截止频率,可通过低频放大倍数和截止频率的乘积计算特征频率:

\[f_T = A_0 f_{cut\: off}\] 

\subsection{操作步骤}

\subsubsection{直流跨导测量}

\begin{enumerate}
    \item 按照 \figref{01} 连接电路,调节 \(E_G\) 和 \(E_D\) 观察对应电压示数,确保电路已经联通。
    \item 首先调节 \(E_D = 6 V\) ,调节 \(E_G = 0.1 V\) 。
    \item 测量 \(V_{GS}\) 和 \(I_D\) 。
    \item 不断调节 \(E_G\) 直到 \(6 V\) 左右\footnote{需要注意,为了 MOS 管正常工作,调节电压时不可使 \(I_D\) 达到 \(300 mA\) 。},重复上述步骤。
    \item 利用差分法进行跨导计算。
\end{enumerate}

\qfig[01]{1001.png}{直流跨导测量电路}

\subsubsection{低频跨导测量}

\begin{enumerate}
    \item 按照 \figref{02} 连接电路,调节 \(E_G\) 和 \(E_D\) 观察对应电压示数,调节任意波形发生器,观察示波器波形,确保电路已经联通。
    \item 调节 \(E_D = 6 V\) ,调节 \(E_G = 4.5 V\) 。
    \item 调节任意波形发生器输出 \(1 kHz\) ,峰-峰值为 \(0.1 V\) 的信号。
    \item 分别记录 \(R_1\) 和 \(R_D\) 上方的交流波形有效值。
    \item 调节任意波形发生器的输出分别为 \(0.3 V\) , \(0.5 V\) , \(0.7 V\) ,\(1 V\) , \(2 V\) ,重复上述步骤。
    \item 利用小信号除法进行跨导计算。
\end{enumerate}

\qfig[02]{1002.png}{低频跨导测量电路}

\subsubsection{交流特性测量}

\begin{enumerate}
    \item 按照上一实验的电路重新搭建,如 \figref{03} 。
    \item 设置 \(E_D = 10 V\)  左右,设置交流信号峰-峰值为 \(100 mV\) ,设置 \(E_G = 3.9 V\) 。
    \item 改变信号的频率,从 \(1 kHz\) 逐渐提升到 \(10 MHz\) ,测量电容上方与 \(R_D\) 上方的波形有效值。 
    \item 将 \(E_G\) 分别设置为 \(4.0 V\),\(4.1 V\) ,\(4.2V\) ,\(4.3V\) ,重复上述步骤。
    \item 计算各个点的跨导 \(g_m\) 和电压增益 \(v_{ds}/v_{gs}\) ,找到截止频率并计算特征频率。
\end{enumerate}

\qfig[03]{1003.png}{频率特性测量电路}

\section{实验数据记录}

原始数据请见 \href{https://github.com/PannenetsF/Mirco-Electronic-Device-Experiment/tree/main/homework/hw10}{这里} 。

\subsection{实验一}

结果如 \figref{04} ,调大负载电阻 \(R_D\) 后曲线如 \figref{05} 。

\qfig[04]{1004.png}{跨导 \(g_m\) 随 \(E_G\) 变化}

\qfig[05]{1005.png}{大负载电阻下跨导 \(g_m\) 随 \(E_G\) 变化}

\subsection{实验二}

如 \figref{06} ,在电压较大时,失真很明显,跨导值不应采用。

\qfig[06]{1006.png}{不同峰-峰值下跨导变化}

\subsection{实验三}

在 \(E_G = 4.2 V\) 下,如 \figref{07}, \figref{08} 。

\qfig[07]{1007.png}{\(E_G = 4.2 V\) 增益}

\qfig[08]{1008.png}{\(E_G = 4.2 V\) 跨导}

\begin{table}[htb]
\centering
\caption{频率特性}
\label{t1}
\begin{tabular}{ccclll}
\toprule
\(E_G\) & \(f_{low}\) & \(A_{low}\) & \multicolumn{1}{c}{\(f_{high}\)} & \multicolumn{1}{c}{\(A_{high}\)} & \(f_T\) \\
\midrule
3.8     & 1k          & 6.93        & 14M                              & 5.08                             & 70M     \\
3.9     & 1k          & 16          & 3M                               & 11.5                             & 35M     \\
4.0     & 1k          & 16.8        & 4.7M                             & 11.7                             & 55M     \\
4.1     & 1k          & 21.2        & 2.1M                             & 15.4                             & 32M     \\
4.2     & 1k          & 24.1        & 2M                               & 17.2                             & 35M    \\ 
\bottomrule
\end{tabular}
\end{table}

\section{实验结果分析}

\subsection{实验一}

在 \(E_G\) 较小时,管子处于截止区,因此跨导极小;在 \(E_G \in (3, 4) V \)  时处于饱和区,此时跨导较大;在 \(E_G\) 进一步增大时,管子进入线性区,此时跨导又一次降低,这导致了跨导的单峰式变化。

由于跨导峰值的变化是和线性区饱和区边界相关的,因此可以通过调节负载电阻进行调节。我们可以将跨导的变化看作是输出电流从 \(0\) 到饱和的迅速变化的一段,由上次实验可以知道,转移特性的陡峭与线性区电流变化的斜率有关,斜率越缓,对应的转移曲线越陡峭,因此可以通过减小负载电阻增大跨导。而受条件所限,通过更换大电阻观察到跨导减小并且范围变窄,得到验证。


\subsection{实验二}

在较小的小信号峰值下,可以得到不失真的波形,并且跨导较为稳定,并且和实验一的数据吻合;在较大的小信号峰值下,大信号会离开饱和区,产生失真,因此跨导不可信。


\subsection{实验三}

首先观察在各个电压下的跨导(由增益和电阻计算而来),与前述实验数据较为吻合;其次在不同的频率下,进行截止频率的测量,在较高的栅极电压下,基本满足稳定的特征频率,而在 \(3.8 V\) 下,由于接近亚阈值区,增益减小,但是频率特性改善,最终特征频率提高\footnote{本次实验采用了 \(3.8-4.2V\) 而不是推荐的 \(3.9-4.3 V\) ,是因为在 \(4.3 V\) 时,采用的晶体管电流过大,可能造成电流过大等问题。}。


\section{总结与思考}

\subsection{问题总结}

由于示波器接到了市电上,会不可避免的收到其电压频率的影响,这也是为什么在经常会见到一个较大的正弦波驮载一个高频正弦波的原因。

\subsection{思考题}

Q1:分析直流跨导随输入电压的变化趋势及其原因。

A1:
在 \(E_G\) 较小时,管子处于截止区,因此跨导极小;在 \(E_G \in (3, 4) V \)  时处于饱和区,此时跨导较大;在 \(E_G\) 进一步增大时,管子进入线性区,此时跨导又一次降低,这导致了跨导的单峰式变化。
由于跨导峰值的变化是和线性区饱和区边界相关的,因此可以通过调节负载电阻进行调节。我们可以将跨导的变化看作是输出电流从 \(0\) 到饱和的迅速变化的一段,由上次实验可以知道,转移特性的陡峭与线性区电流变化的斜率有关,斜率越缓,对应的转移曲线越陡峭,因此可以通过减小负载电阻增大跨导。而受条件所限,通过更换大电阻观察到跨导减小并且范围变窄,得到验证。

Q2: 电流源内阻和负载电阻R D 对转移特性曲线和跨导具有什么影响?

A2: 由 Q2 的分析,在漏极的总电阻实际可以看作一个 \(R_D'\) ,\(R_D'\) 越大,转移特性曲线越陡峭,大跨导范围越小,峰值也会减小; \(R_D'\) 越小,大跨导范围越大,峰值也会上升。
% Start Here

% End Here

\end{document}