\documentclass[lang=cn,11pt,a4paper,cite=authoryear]{elegantpaper}

\input{needed.tex}

\title{微电子器件实验\quad 二极管的直流特性测量与分析}
\author{范云潜, 学号:18373486,搭档:徐靖涵,教师:彭守仲}
\institute{微电子学院 184111 班}
\date{\zhdate{2020/10/19}}

\begin{document}

\maketitle

% \tableofcontents

\section{实验目的}

测量二极管的直流特性并进行分析,并从中:

\begin{itemize}
    \item 了解通用仪表的的基本原理和使用方法
    \item 了解被测器件各项参数的定义和测量方法
    \item 掌握被测器件直流特性和相关机理
\end{itemize}

\section{实验所用设备及器件}

实验用到的核心器件是二极管,其他设备包括面包板、杜邦线、电压源、手持万用表、台式万用表等。

\section{实验基本原理及步骤}

基本原理是基于二极管的单向导通特性连接电路后,在不同电压下读取对应的电流,来获得对应的直流特性。

步骤:\begin{enumerate}
    \item 搭建电路,电路图如\figref{02}。
    \item 正向导通时,电压间隔为 \(0.1 V\) ,记录二极管两端的电压与内部电流
    \item 反向导通时, \(0 - \ -1 V\) 间隔为 \(0.1 V\) ,\(-1 - -5V\) 间隔为 \(0.5 V\) ,\(-5 - -10 V\) 间隔为 \(1 V\) ,\(-10 - 30 V\) 间隔为 \(5 V\) ,记录电流
    \item 拆解电路,还原仪器
\end{enumerate}



\qfig[02]{0701p2.png}{测试电路}

\section{实验数据记录}

原始数据可视化后如\figref{01} ,原始数据请见 \href{https://github.com/PannenetsF/Mirco-Electronic-Device-Experiment/tree/main/homework/hw07/01led}{这里} 。

\qfig[01]{0701p1.png}{LED \(I-V\) 曲线}

\section{实验结果分析}

在得到的直流特性曲线 \figref{01} 中可以看出,在正向导通时,电流随电压成指数变化,较小电压变化会引起较大的电流变化;反向导通且未击穿时,较大的电压变化引起的电流增大较小。

\section{总结与思考}

Q:在二极管直流特性测量中,应该采用电流表外接法还是电流表内接法?为什么?

A:电流表外接法较为准确的条件是电流表分压的影响大于电压表分流的影响,也就是被测器件的电阻较小;内接法相反,被测器件的电阻需要较大。在正向导通时,电阻较小,应该采取外接法;反向导通时,电阻较大,应该采取内接法。

% Start Here

% End Here

\end{document}