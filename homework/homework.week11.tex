\documentclass[lang=cn,11pt,a4paper,cite=authoryear]{elegantpaper}

% 微分号
\newcommand{\dd}[1]{\mathrm{d}#1}
\newcommand{\pp}[1]{\partial{}#1}

\newcommand{\homep}[1]{\section*{Problem #1}}

% FT LT ZT
\newcommand{\ft}[1]{\mathscr{F}[#1]}
\newcommand{\fta}{\xrightarrow{\mathscr{F}}}
\newcommand{\lt}[1]{\mathscr{L}[#1]}
\newcommand{\lta}{\xrightarrow{\mathscr{L}}}
\newcommand{\zt}[1]{\mathscr{Z}[#1]}
\newcommand{\zta}{\xrightarrow{\mathscr{Z}}}

% 积分求和号

\newcommand{\dsum}{\displaystyle\sum}
\newcommand{\aint}{\int_{-\infty}^{+\infty}}

% 简易图片插入
\newcommand{\qfig}[3][nolabel]{
  \begin{figure}[!htb]
      \centering
      \includegraphics[width=0.6\textwidth]{#2}
      \caption{#3}
      \label{#1}
  \end{figure}
}

% 表格
\renewcommand\arraystretch{1.5}


% 日期


\title{微电子器件实验\quad 模型参数测量}
\author{范云潜, 学号:18373486,搭档:徐靖涵,教师:彭守仲}
\institute{微电子学院 184111 班}
\date{\zhtoday}

\begin{document}

\maketitle

% \tableofcontents

\section{实验目的}

对双极型晶体管以及场效应管进行模型参数测量,获得其跨导,长度调制系数等关键参数。

\section{实验所用设备及器件}


主要设备有:电压源,任意波形发生器,示波器,手持式万用表,台式万用表,相关线缆等,主要器件有 晶体管 C9018 与 K656 。

\section{实验基本原理及步骤}

\subsection{基本原理}

在前序课程中,我们了解到,晶体管的放大特性体现在小信号上,也就是在静态工作点附近变化的电学信号上,而这些信号的约束条件是由电学关系的微分提供的,比较关键的是跨导和沟道调制系数。接下来,我们将在小信号模型上对这些约束关系进行推导。

\subsubsection{双极型晶体管小信号等效电路}

原始电路如 \figref{01}, 小信号电路如 \figref{02} 。

\qfig[01]{1103.png}{双极型晶体管测量电路}

\qfig[02]{1101.png}{双极型晶体管测量电路等效小信号电路}

对小信号模型进行约束: 

\[\begin{aligned}
    (v_{be} - v_i) / R_i + i_b + \dfrac{v_{be}}{R_B} &= 0 \\ 
    \beta i_b (R_c // R_2) &= - v_o \\ 
    i_b r_\pi &= v_{be} \\ 
    v_o &= v_{ce} \\
\end{aligned}\]

解得:

\[r_\pi = v_{be} / (-\dfrac{v_{be}}{R_B} - \dfrac{v_{be} - v_i}{R_i}) \] 

\[\beta = \dfrac{-v_{ce}}{R_c // R_2 (-\dfrac{v_{be}}{R_B} - \dfrac{v_{be} - v_i}{R_i})}\]

\[A_v = \dfrac{v_o}{v_i} = - \dfrac{\beta (R_c // R_2)}{r_\pi + R_i + r_\pi R_i / R_B}\]

\[g_m = \dfrac{i_c}{v_i} = \beta \dfrac{(-\dfrac{v_{be}}{R_B} - \dfrac{v_{be} - v_i}{R_i})}{v_i}\]


\subsubsection{场效应管小信号等效电路}

原始电路如 \figref{03}, 小信号电路如 \figref{04} 。

\qfig[03]{1104.png}{场效应管测量电路}

\qfig[04]{1102.png}{场效应双极型晶体管测量电路等效小信号电路}


对小信号模型进行约束: 

\[\begin{aligned}
    - v_{ds1} &= g_m v_{gs1} (r_{ds} // R_D // R_{L1}) \\ 
    - v_{ds2} &= g_m v_{gs2} (r_{ds} // R_D // R_{L2}) 
\end{aligned}\]

做比: 

\[\frac{v_{ds1}}{v_{ds2}} = \frac{v_{gs1}}{v_{gs2}} \frac{R_D // R_{L1}}{R_D // R_{L2}} \frac{r_{ds} + R_D // R_{L2}}{r_{ds} + R_D // R_{L1}}\]

进行数据带入: \[\begin{aligned}
    \frac{v_{ds1}}{v_{ds2}} &= \frac{v_{gs1}}{v_{gs2}} \frac{k_1}{k_2} \frac{r_{ds} + k_2}{r_{ds} + k_1} \\
    r_{ds} &= \dfrac{1-A_1/A_2}{\dfrac{A_1 k_2}{A_2 k_1} - 1}
\end{aligned}, \text{ where } A = \frac{v_{ds}}{v_{gs}}\]

带入得到 \[g_m = - \frac{A_1}{r_{ds} // R_D // R_{L1}}\]  

\[\lambda = \frac{1}{I_D r_{ds}}\]

\subsection{基本步骤}

\subsection{实验一} 

\begin{enumerate}
    \item 按照实验电路 \figref{01} 搭建电路
    \item 调节 \(E_C = 15 V\) ,调节 \(E_B\) 使得 \(I_B = 60 \mu A\) 
    \item 调节任意波形发生器,使之输出 \(1 kHz, 7.5 V \) 的电压 \(v_b\) 
    \item 通过示波器测量小信号电压 \(v_{be}\) 和 \(v_{ce}\) 的 RMS 
\end{enumerate}

\subsection{实验二} 

首先进行直流工作点的寻找:

\begin{enumerate}
    \item 按照 \figref{05} 搭建电路
    \item 调节 \(E_G\) 到 \(4.5 V\) 左右,调节 \(E_D\) 到 \(30 V\) 左右
    \item 绘制关于 \(E_G\) 的电流变化,确保工作在恒流区
\end{enumerate}

\qfig[05]{1105.png}{场效应管直流工作点测试电路}

之后进行实验电路搭建: 

\begin{enumerate}
    \item 按照实验电路 \figref{03} 搭建电路
    \item 保持上一步骤的直流工作点,调节任意波形发生器,使之输出 \(1 kHz, 0.1 V \) 的电压 \(v_g\) 
    \item 将 \(R_L\) 分别改为 \(1 k\Omega , 6.8 k\Omega\),记录小信号 \(v_{gs}\) 和 \(v_{ds}\) 的 RMS 
\end{enumerate}

\section{实验数据记录}


原始数据请见 \href{https://github.com/PannenetsF/Mirco-Electronic-Device-Experiment/tree/main/homework/hw11}{这里} 。

\subsection{实验一}

测量数据以及计算结果如 \tabref{t1} 。

\begin{table}[htb]
\centering
\caption{双极型晶体管特性测量}
\label{t1}
\begin{tabular}{|c|c|c|c|c|c|c|}
\hline
IB       & EC  & EB   & VCE      & vbe      & vce      & rb     \\ \hline
6.01E-05 & 15  & 6.7  & 0.26     & 1.40E-02 & 1.51E-01 & 100000 \\ \hline
r1       & r2  & rc   & beta     & gm       & rp       & av     \\ \hline
100000   & 100 & 1000 & 9.42E+01 & 0.1186   & 7.94E+02 & 0.0854 \\ \hline
\end{tabular}
\end{table}

\subsection{实验二}

通过绘制转移图如 \figref{06} ,确定 \(E_G = 4.5 V\) 时,晶体管工作在饱和区,测量数据以及计算结果如 \tabref{t2} 。


\qfig[06]{1106.png}{转移曲线}

\begin{table}[htb]
\centering
\caption{场效应管特性测量}
\label{t2}
\begin{tabular}{|c|c|c|c|c|c|c|c|c|c|c|c|c|}
\hline
EG  & ED & VGS   & VDS  & ID     & RD & RL   & RG   & vgs    & vds  & rds  & gm    & lamb   \\ \hline
4.5 & 30 & 0.807 & 6.82 & 0.0234 & 1k & 1k   & 6.8k & 0.0294 & 2.32 & 4594 & 0.175 & 0.0093 \\ \hline
4.5 & 30 & 0.807 & 6.82 & 0.0234 & 1k & 6.8k & 6.8k & 0.0294 & 3.77 &      &       &        \\ \hline
\end{tabular}
\end{table}

\section{实验结果分析}

由于电压是实验者在小信号电路中唯一可以测量的量,因此需要将各种特性和电压建立关系,进一步进行求解。

\section{总结与思考}

Q1:双极型三极管的直流电阻 \(R_{BE}\) 和 \(R_{CE}\) 如何变化?

A1:直流条件下根据数据观察可得, \(I_C\) 越大 ,\(V_{CE}\) 越小,那么 \(R_{CE}\) 越小; \(V_{BE}\) 几乎不随 \(I_C\) 变化,那么 \(R_{BE}\) 也是变小的。

Q2:如何测量双极型晶体管小信号模型中的电阻\(r_{ce}\)?
根据之前测量的输出特性曲线计算 \(r_{ce}\) 的大小。

如果在本次实验基础上进行修改,只需要更换 \(R_2\) 阻值重新测量,联立即可。根据之前的数据\footnote{第八周实验的放大区数据}, \(r_{ce} = v_{ce} / i_o \approx 16 k\Omega\) 。

Q3:MOS管的直流电阻 \(R_{GS}\) 和 \(R_{DS}\) 如何变化 ?

\(I_D\) 越大 ,\(V_{DS}\) 越小,那么 \(R_{DS}\) 越小;由于栅极无电流通过,因此电阻不变,恒为开路。

Q4: \(R_L\) 阻值变化后,低频跨导 \(g_m\) 是否变化?为什么?

不变,由于跨导可以看作直流特性的导数,而 \(R_L\) 不影响直流特性。

% Start Here

% End Here

\end{document}