\documentclass[lang=cn,11pt,a4paper,cite=authoryear]{elegantpaper}

\input{needed.tex}

\title{微电子器件实验\quad 双极型晶体管的直流特性测量与分析}
\author{范云潜, 学号:18373486,搭档:徐靖涵,教师:彭守仲}
\institute{微电子学院 184111 班}
\date{\zhdate{2020/10/21}}

\begin{document}

\maketitle

% \tableofcontents

\section{实验目的}

测量不同基极电压情况下,输出电流随着发射极电压的变化情况,并绘制对应的直流特性曲线。

\section{实验所用设备及器件}

实验用到的核心器件是双极型晶体管 9018 ,其他设备包括面包板、杜邦线、电压源、手持万用表、台式万用表、电阻等。


\section{实验基本原理及步骤}

在不同基极电流情况下,随着集电极电压的变化,电流变化可以依次分为截止区、饱和区与放大区。在放大区, \(I_C\) 是对 \(I_B\) 的成倍数放大;饱和区电流会对电压饱和,如\figref{01} 。

\qfig[01]{0703p1.png}{输出特性原理}

实验步骤:

\begin{enumerate}
    \item 搭建电路如\figref{02}
    \item 调节 \(V_B\) 使得 \(I_B = 20 \mu A\) 
    \item 调节 \(E_C\) 分别在 \(0.1-1V\) 以 \(0.1V\) 为间隔,在 \(1-10V\) 以 \(1V\) 为间隔,测量 \(V_{CE}\) 与 \(I_C\) 
    \item 调节 \(V_B\) 使得 \(I_B = 40/60/80/100 \mu A\) ,重复上述步骤
    \item 关闭电源,拆解电路,恢复仪器。
\end{enumerate}

\qfig[02]{0703p2.png}{实验电路图}

\section{实验数据记录}


原始数据可视化后如\figref{03} ,原始数据请见 \href{https://github.com/PannenetsF/Mirco-Electronic-Device-Experiment/tree/main/homework/hw07/02bjt}{这里} 。

\qfig[03]{0703p3.png}{实验结果 \(I_{C}-V_{CE}\) 图}

\section{实验结果分析}

在不同的电压 \(V_{CE}\) 下电流随之上升,分别经过饱和区与放大区。在不同的基极电流 \(I_B\)下,放大区电流可以看出较为线性的放大效果,如 \figref{04}。

\qfig[04]{0703p4.png}{实验结果 \(I_{C}-I_{B}\) 图}

\section{总结与思考}

Q1:截止区、放大区、饱和区的特点?

截止区时发射结几乎未导通,集电极无电流;放大区时发射结正偏,集电结反偏,\(I_C\) 几乎与 \(V_{CE}\) 无关,是对 \(I_B\) 的放大;饱和区发射结与集电结均正偏,\(V_{CE}\) 较小,\(I_C\) 随 \(V_{CE}\) 变化较大,\(I_B\) 的变化几乎不会引起 \(I_{C}\) 变化。

Q2:当 \(V_{CE}\) 增大时, \(I_B\) 如何变化,机理是什么?

首先经过饱和区,\(I_B\) 急剧上升;再是放大区,变化较为平缓,但是由于基区调制效应,并不是严格的曲线。

% Start Here

% End Here

\end{document}