\documentclass[lang=cn,11pt,a4paper,cite=authoryear]{elegantpaper}

\input{needed.tex}

\title{微电子器件实验\quad xxxxxxxx}
\author{范云潜, 学号:18373486,搭档:徐靖涵,教师:彭守仲}
\institute{微电子学院 184111 班}
\date{\zhtoday}

\begin{document}

\maketitle

% \tableofcontents

\section{实验目的}

\section{实验所用设备及器件}

\section{实验基本原理及步骤}


\subsection{BJT 双极型晶体管测量}

曲线测量涉及到多个点的数据,因此不可避免的使用到循环结构。本次实验中,所有电压设置为等间距采样,同时除去扫描电压外,还需对基极偏置进行调整,这对应另一层循环。大体流程可以写成如~算法\ref{alg:B}的伪代码,并将其转换成如\figref{l3d}的流程图,其前面板如\figref{l3p},得到结果如\figref{l3p},可以看到与之前实验结果符合较好。

\qfig[l3d]{lab3d.png}{BJT 测量电路后面板}

\qfig[l3p]{lab3p.png}{BJT 测量电路前面板}

% \qfig[l3a]{lab3a.png}{BJT 测量结果}

\begin{algorithm}
    \caption{BJT 特性曲线测量流程}
    \label{alg:B}
    \begin{algorithmic}
        \STATE{set minVc, stepVc, maxVc} 
        \STATE{set minVb, stepVb, maxVb} 
        \STATE{get \# stepC = (maxVc - minVc) / stepVc}
        \STATE{get \# stepB = (maxVb - minVb) / stepVb}
        \STATE{set i = 0, j = 0}
        \STATE{set Vc, Ic as array2d}
        \STATE{set Vct, Ict as array}
            \REPEAT
                \STATE{VoutB = i * stepVb + minVb}
                \STATE{i += 1}
                \STATE{set VoutB to 2231A}
                \REPEAT 
                    \STATE{VoutC = j * stepVc + minVc}
                    \STATE{set VoutC to 2231A}
                    \STATE{get Iout from DMM6500}
                    \STATE{Vct = [Vct, VoutC]}
                    \STATE{Ict = [Ict, Iout]}
                    \STATE{j += 1}
                    \UNTIL{j = \# stepC}
                \STATE{Vc = [Vc, Vct]}
                \STATE{Ic = [Ic, Ict]}
                \STATE{set filename = 'data\_ new\_ ' + num2string(i) + '.txt'} 
                \STATE{open file = filename}
                \STATE{set title = 'VC(V)             IC(A)' + newline()}
                \STATE{write title to file}
                \STATE{set j = 0}
                    \REPEAT
                        \STATE{set toWrite = num2string(Vct[j]) + ' ' + num2string(Ict[j]) + newline()}
                        \STATE{write toWrite to file}
                        \STATE{j += 1}
                    \UNTIL{j = \# stepC}
                \STATE{clear Ict, Vct}
                \STATE{close file}
        \UNTIL{i == \# stepB }
        \STATE{get graph of [Vc, Ic]}
    \end{algorithmic}
\end{algorithm}



\section{实验数据记录}

\section{实验结果分析}

\section{总结与思考}

% Start Here

% End Here

\end{document}