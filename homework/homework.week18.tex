\documentclass[lang=cn,11pt,a4paper,cite=authoryear]{elegantpaper}

% 微分号
\newcommand{\dd}[1]{\mathrm{d}#1}
\newcommand{\pp}[1]{\partial{}#1}

\newcommand{\homep}[1]{\section*{Problem #1}}

% FT LT ZT
\newcommand{\ft}[1]{\mathscr{F}[#1]}
\newcommand{\fta}{\xrightarrow{\mathscr{F}}}
\newcommand{\lt}[1]{\mathscr{L}[#1]}
\newcommand{\lta}{\xrightarrow{\mathscr{L}}}
\newcommand{\zt}[1]{\mathscr{Z}[#1]}
\newcommand{\zta}{\xrightarrow{\mathscr{Z}}}

% 积分求和号

\newcommand{\dsum}{\displaystyle\sum}
\newcommand{\aint}{\int_{-\infty}^{+\infty}}

% 简易图片插入
\newcommand{\qfig}[3][nolabel]{
  \begin{figure}[!htb]
      \centering
      \includegraphics[width=0.6\textwidth]{#2}
      \caption{#3}
      \label{#1}
  \end{figure}
}

% 表格
\renewcommand\arraystretch{1.5}


% 日期


\usepackage{algorithm}
\usepackage{algorithmic}

\title{微电子器件实验\quad 大作业}
\author{范云潜, 学号:18373486,搭档:徐靖涵,教师:彭守仲}
\institute{微电子学院 184111 班}
\date{\zhtoday}

\begin{document}

\maketitle

% \tableofcontents

\section{实验目的}


利用 LabVIEW 软件对 LED 发光二极管、BJT 双极型晶体管等进行自动的直流电流-电压特性测量,了解在实际实验中电子元器件的电学测量自动化方法,并且掌握基本的自动化测试方法,建立一定的测试思维。

\section{实验所用设备及器件}


主要设备有:电压源,台式万用表,相关线缆等,主要器件有双极型晶体管,软件为 LabVIEW 2020。

\section{实验基本原理及步骤}

曲线测量涉及到多个点的数据,因此不可避免的使用到循环结构。本次实验中,所有电压设置为等间距采样,同时除去扫描电压外,还需对基极偏置进行调整,这对应另一层循环。大体流程可以写成如~算法\ref{alg:B}的伪代码,并将其转换成如\figref{l3d}的流程图,其前面板如\figref{l3p},得到结果如\figref{l3p},可以看到与之前实验结果符合较好。

过程的思路进行如下简述。本次测量需要在不同基极电压下扫描集电极电压,得到响应的电流,绘制数条 \(I-V_{C}\) 图线,这类电压遍历需要用到双层的循环结构,同时绘图需要输入是二维数组,每次循环得到一个一维数组,因此需要将不同次的数组合并成同一个二维数组。为了合并这个数组,需要保证每次的数组大小是一致的,这可以通过循环的步长决定,但是由于 LabVIEW 不支持局部变量,因此需要在每次循环完成后进行手动的清零。在完成二维数组获取的工作后,需要进行画图与数据存储。画图有内置的 XY 绘图仪支持对二维数组的绘图,将不同维度设置成图像的通道。数据存储涉及到二维数组的拆分,为了减少这部分合并再拆分的工作量,在生成一维数组时完成数据的存储即可。根据仪器说明知道,存储需要的文件名数据类型是 FileName ,因此将数据的索引、偏压等转储到字符串后还需要进行到 FileName 的转换。另一方面,数据转换成字符串时需要注意添加空格等。

\qfig[l3d]{lab3d.png}{BJT 测量电路后面板}

\qfig[l3p]{lab3p.png}{BJT 测量电路前面板}

% \qfig[l3a]{lab3a.png}{BJT 测量结果}

\begin{algorithm}
    \caption{BJT 特性曲线测量流程}
    \label{alg:B}
    \begin{algorithmic}
        \STATE{set minVc, stepVc, maxVc} 
        \STATE{set minVb, stepVb, maxVb} 
        \STATE{get \# stepC = (maxVc - minVc) / stepVc}
        \STATE{get \# stepB = (maxVb - minVb) / stepVb}
        \STATE{set i = 0, j = 0}
        \STATE{set Vc, Ic as array2d}
        \STATE{set Vct, Ict as array}
            \REPEAT
                \STATE{VoutB = i * stepVb + minVb}
                \STATE{i += 1}
                \STATE{set VoutB to 2231A}
                \REPEAT 
                    \STATE{VoutC = j * stepVc + minVc}
                    \STATE{set VoutC to 2231A}
                    \STATE{get Iout from DMM6500}
                    \STATE{Vct = [Vct, VoutC]}
                    \STATE{Ict = [Ict, Iout]}
                    \STATE{j += 1}
                    \UNTIL{j = \# stepC}
                \STATE{Vc = [Vc, Vct]}
                \STATE{Ic = [Ic, Ict]}
                \STATE{set filename = 'data\_ new\_ ' + num2string(i) + '.txt'} 
                \STATE{open file = filename}
                \STATE{set title = 'VC(V)             IC(A)' + newline()}
                \STATE{write title to file}
                \STATE{set j = 0}
                    \REPEAT
                        \STATE{set toWrite = num2string(Vct[j]) + ' ' + num2string(Ict[j]) + newline()}
                        \STATE{write toWrite to file}
                        \STATE{j += 1}
                    \UNTIL{j = \# stepC}
                \STATE{clear Ict, Vct}
                \STATE{close file}
        \UNTIL{i == \# stepB }
        \STATE{get graph of [Vc, Ic]}
    \end{algorithmic}
\end{algorithm}

\section{实验过程总结}

实验过程整体思路是清晰的,只有一个小点较为费解,就是在双重 For Loop 中每一层接连为电压源赋值。在实验中,我们的实现方法是,如果 \(V_B\) 需要更新,那么这次就只更新其对应的 Channel ,因为 \(V_B\) 在一次外循环中仅仅在开头更新一次,那么此时 \(V_C\) 是上一次循环的最大值,出现折返的问题,因此需要后续将其切除。

在结束后,我思考得到另一种解决方案,建立一个单数据 Indicator ,在程序开始为其初始化,并将其连接到电压源的电压控制上,在 \(V_B\) 循环开始,为其赋值 \(V_{Bt}\) ,在 \(V_C\) 循环内,为其赋值 \(V_{Ct}\) ,就可避免以上问题。其实遇到很多问题都是这个原因:编程思想在这种直观但是表达方式欠佳的方式中不知道如何体现。

大部分时间是消耗在查文档上,也就是想办法去实现自己的想法,简单的思路但是实现起来不一定直观。我和队友对这种直观方法与之前接触的偏向硬件或者对象的编程语言如 Python , Scala 等进行对比,发现对于编程的方式,思路的实现较为灵活,而这种直观的方式需要手动连线,这要求操作者较为熟练地掌握需要的虚拟仪器;当出现错误的时候,编程的方式难以直观的映射到对应的仪器上,而手动连线更为方便。因此我认为,可能之后会出现一种将编程语言翻译成对应的测试电路的方式,就如同其他硬件设计、测试领域一样。

\section{课程回顾、思考与感想}

整个课程循序渐进,比较成体系,很锻炼动手能力。

这学期课程的主要内容是在\textbf{实验条件}下测量、计算相关元器件的电学特性,其中需要利用到与理论课程相关的公式,来计算一些特性,如伏安特性曲线、放大系数、跨导等,进行了多次的实践和设计,并且需要对实验结果和理论分析作出相应的对比,来分析实验结果的正确性,从而加深对理论知识的认识。

另一方面,也可以更深入的认识到理论的诸多限制,如输出电压饱和,这是在理论学习中直接使用平方律公式无法直观感受到的。

同时,在日常测量中的机械、重复性的工作也让人印象深刻,若是更复杂的工作仅靠人工测量会消耗大量的人力物力,并且精度不能保证,这就引入了测量自动化的必要性。

但是在日常的实验中,会有一些比较\textbf{奇奇怪怪}的小问题,电路的拓扑没有问题,但是可能会出现一些波形振荡的问题,最终通过电路重插、换孔等方式解决,预估是出现了接触不良,我认为这部分时间在理论上没有更多的效用,但是在实际操作中的影响是相当巨大的。这也激励我更规范、更合理的规划整个器件集合的摆放。相同问题在其他涉及到硬件的问题中也是通用的。

最后感谢老师和几位助教一学期的指导!

\begin{flushleft}
    祝好!
    范云潜敬上
\end{flushleft}


\end{document}