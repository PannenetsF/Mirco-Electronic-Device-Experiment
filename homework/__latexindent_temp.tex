\documentclass[lang=cn,11pt,a4paper,cite=authoryear]{elegantpaper}

\input{needed.tex}

\title{微电子器件实验\quad MOS 管直流特性}
\author{范云潜, 学号:18373486,搭档:徐靖涵,教师:彭守仲}
\institute{微电子学院 184111 班}
\date{\zhdate{2020/10/19}}

\begin{document}

\maketitle

% \tableofcontents

\section{实验目的}

从场效应管的直流特性特性的测量与分析中,验证并加深对场效应管原理与性质的理解。

\section{实验所用设备及器件}

主要设备有:电压源,手持式万用表,台式万用表,相关线缆等,主要器件有 N 沟道 MOS 管 IRF3205 和 IRFR214 。

\section{实验基本原理及步骤}

\subsection{MOS 管基本结构}

场效应管(Field Effect Transistor, FET)是一种压控电流元件,仅靠多子进行导电,又称为单极型晶体管,并且体积小、重量轻、寿命长、噪音低、热稳定性好、耗电低是现代的超大规模数字集成电路(Very Large Scale Integration, VLSI)的基本器件。

FET 主要可以分为结型场效应管(Junction Field Effect Transistor, JFET) 和绝缘栅型场效应管(Insulated Gate Effect Transistor, IGFET)。后者有常见的金属-氧化物-半导体结构,也就是所谓的 MOS 管。 MOSFET 有增强型和耗尽型两类器件,又可以分别分为 N 沟道与 P 沟道。

NMOS 的基本结构如 \figref{01} ,一个低掺杂的P型硅片为衬底,两个高掺杂的 N型阱区引出电极作为源极和漏极,覆盖一层二氧化硅绝缘层后再覆盖一层金属铝引出电极作为栅极。

\qfig[01]{09p1.png}{NMOS 基本结构}

\subsection{NMOS 电学特性} 

在栅极-源极不外加电压 \(V_{GS} = 0\) 时, 源漏之间是两个 PN 结,不存在导电沟道,因此 \(V_{DS} \neq 0\) 时也不会有电流通过;在 \(V_{GS} > 0\) 时,电压透过绝缘层作用到栅极下方,排斥其中的空穴,留下电离的杂质负离子形成了耗尽层,但是由于没有自由移动的载流子,因此仍然无法形成电流;当 \(V_{GS}\) 进一步增大,已经不存在可以排斥的空穴,因此吸引来了电子,形成了 N 型的层,也就是反型层,构成了导电的沟道。产生导电沟道的临界电压称为阈值电压 \(V_{th}\) 。

当 \(V_{GS} > V_{th}\) 时,源漏电压可以引起漏极电流。当 \(V_{DS}\) 较小时,未产生夹断,并且 \(I_D\) 随着 \(V_{DS}\) 线性变化,称为线性区;当 \(V_{DS}\) 较大时,沟道出现了夹断,进一步增大会造成夹断区的延长,而夹断区的长度变化几乎不会影响电流,电流几乎仅取决于 \(V_{GS}\) ,此时称为饱和区。

对上述的直流特性进行总结,可以得到 \figref{02} 。

\qfig[02]{09p2.png}{NMOS 输出特性曲线}

由于工作在饱和区时,不同的 \(V_{DS}\) 输出电流几乎相同,可用转移特性曲线进行描述饱和区的电流特性,如 \figref{03} 。

\qfig[03]{09p3.png}{NMOS 转移特性曲线}

\subsection{操作步骤}

\subsubsection{实验一:IRF3205 源漏伏安特性曲线}

\begin{enumerate}
    \item 将元器件连接到面包板,尽量减少杜邦线的使用。电路如 \figref{04} 
    \item 连线完成后打开 \(E_D\) 测试仪表是否工作正常
    \item 调节 \(E_D\) 为 \(0-1.5 V\) 
    \item 测量对应的 \(V_{DS}\) 和  \(I_D\) 并绘图
\end{enumerate}

\qfig[04]{09p4.png}{实验一电路}

\subsubsection{实验二:IRF3205 转移特性曲线}


\begin{enumerate}
    \item 将元器件连接到面包板,尽量减少杜邦线的使用。电路如 \figref{05} 。
    \item 连线完成后打开 \(E_D\) 和 \(E_G\) 测试仪表是否工作正常
    \item 调节 \(E_D\) 为 \(0.5, 1, 1.5 V\) 
    \item 分别将 \(E_G\) 在 \(0.1 - 7 V\) 取值
    \item 测量对应的 \(V_{GS}\) 和  \(I_D\) 并绘图
\end{enumerate}

\qfig[05]{09p5.png}{实验二电路}

\subsubsection{实验三:IRFR214 转移特性曲线}

\begin{enumerate}
    \item 将元器件连接到面包板,尽量减少杜邦线的使用。电路如 \figref{06} 。
    \item 连线完成后打开 \(E_D\) 和 \(E_G\) 测试仪表是否工作正常
    \item 调节 \(E_D\) 为 \(1.0 V\) 
    \item 分别将 \(E_G\) 在 \(0.1 - 10 V\) 取值
    \item 测量对应的 \(V_{GS}\) 和  \(I_D\) 并绘图
\end{enumerate}

\qfig[06]{09p6.png}{实验三电路}


\subsubsection{实验四:IRFR214 输出特性曲线}


\begin{enumerate}
    \item 将元器件连接到面包板,尽量减少杜邦线的使用。电路如 \figref{07} 。
    \item 连线完成后打开 \(E_D\) 和 \(E_G\) 测试仪表是否工作正常
    \item 调节 \(E_G\) 为 \(3.3, 3.5, 3.7 V\) 
    \item 分别将 \(E_D\) 在 \(0.1 - 2 V\) 取值
    \item 测量对应的 \(V_{DS}\) 和  \(I_D\) 并绘图
\end{enumerate}

\qfig[07]{09p6.png}{实验四电路}

\section{实验数据记录}

原始数据请见 \href{https://github.com/PannenetsF/Mirco-Electronic-Device-Experiment/tree/main/homework/hw09}{这里} 。

\subsection{实验一}

IRF3205 源漏伏安特性曲线如\figref{08} 。


\qfig[08]{09p7.png}{IRF3205 源漏伏安特性曲线}


\subsection{实验二}

IRF3205 转移特性曲线如\figref{09} 。


\qfig[09]{09p8.png}{IRF3205 转移特性曲线}


\subsection{实验三}

IRFR214 转移特性曲线如\figref{10} 。


\qfig[10]{09p9.png}{IRFR214 转移特性曲线}


\subsection{实验四}

IRF3205 源漏伏安特性曲线如\figref{11} 。


\qfig[11]{09p10.png}{IRFR214 输出特性曲线}


\section{实验结果分析}

\subsection{实验一} 

在未外加栅源电压时,源漏之间可以看作是一个二极管,因此存在一个阈值以及相似的电流特性。从 \figref{08} 可以看出,阈值约为 \(0.6V\) 并且满足指数上升的特点。

\subsection{实验二}\label{sec1}

在 \(V_{GS}\) 较小时,器件处于截止状态,因此电流较小,不同的 \(V_D\) 对电流影响较小;稍大的 \(V_{GS}\) 使得器件工作在饱和状态, \(V_{GS}\) 对电流的影响是二次的,不同 \(V_{DS}\) 在可变电阻区的影响体现在饱和区的入口,也就是为什么同一 \(V_{GS}\) 下电流的大小不同;进一步增大,由于栅极电压过高,管子工作在线性区,\(V_{GS}\) 对电流的影响较小,因此很小的电流变化就涵盖了较大的 \(V_{GS}\) 区域。

\subsection{实验三}

同 \ref{sec1} 。

\subsection{实验四}

在 \(E_D\) 较小时,电路处在线性区,电流较小; \(E_D\) 增大时,逐渐进入线性区,电流迅速上升;继续增大,进入饱和区,电流几乎保持不变。较小的栅极偏置差距下,电流差距变化迅速,体现了平方增长的特点。

\section{总结与思考}

转移特性曲线变平的原因:此时由于栅极电压过高,管子工作在线性区,\(V_{GS}\) 对电流的影响较小,因此很小的电流变化就涵盖了较大的 \(V_{GS}\) 区域。

Q1. 请列举场效应管与双极型晶体管的几个不同点

\begin{itemize}
    \item MOS 是压控器件,BJT 是流控器件,因此 MOS 的功耗更低
    \item MOS 仅有多子参与导电, BJT 多子少子均参与,因此MOS 的温度性更好
    \item MOS 集成性好
\end{itemize}

Q2. 请根据转移特性曲线图画出输出特性曲线中的恒流区曲线示意图。

如 \figref{ans} 。

\qfig[ans]{09p11.png}{恒流区曲线}


% Start Here

% End Here

\end{document}