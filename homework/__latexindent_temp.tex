\documentclass[lang=cn,11pt,a4paper,cite=authoryear]{elegantpaper}

\input{needed.tex}
\usepackage{longtable}

\title{微电子器件实验\quad 双极型晶体管的直流特性和交流特性测量和分析}
\author{范云潜, 学号:18373486,搭档:徐靖涵,教师:彭守仲}
\institute{微电子学院 184111 班}
\date{\zhdate{2020/10/19}}

\begin{document}

\maketitle

% \tableofcontents

\section{实验目的}

对双极型晶体管的直流特性和交流特性进行测量和分析,验证其放大特性。

\section{实验所用设备及器件}


实验用到的核心器件是双极型晶体管 9018 ,其他设备包括面包板、杜邦线、电压源、手持万用表、台式万用表、电阻、电容等。

\section{实验基本原理及步骤}

三极管工作在放大区时,可以作为放大器使用。放大器作用的是小信号,因此需要寻找对应的静态工作点。

步骤如下 :

\begin{enumerate}
    \item 首先搭建直流工作电路,如 \figref{01} 。调节 \(E_B\) 使得 \(I_B = 60 \mu A \) 。
    \item 调节 \(E_C\) 从 \(0.1 V\) 到 \(20 V\) 变化,记录 \(I_C - V_{CE}\) 的数据并画图。
    \item 在 \(E_C = 15 V\) 的条件下,进行交流特性的测量。修改电路如 \figref{02} 。
    \item 使用任意波形发生器输出 \(1 kHz\) 的信号,峰峰电压分别为 \(1 V, 2.5 V, 5 V, 7.5 V, 10 V, 15 V\) 。
    \item 分别测量 \(R_1\) 和 \(R_2\) 的电压波形,计算电流放大系数 \(i_C/i_B\) 。
    \item 任意波形发生器的电压设置为 \(10 V\) ,改变输出频率,测量 \(R_2\) 的电压波形和有效值,计算电流的放大系数,并且寻找三分贝带宽 \(f_\beta\) ,计算特征频率。
\end{enumerate}

\qfig[01]{08p1.png}{直流工作电路}

\qfig[02]{08p2.png}{交流工作电路}

\section{实验数据记录}

原始数据请见 \href{https://github.com/PannenetsF/Mirco-Electronic-Device-Experiment/tree/main/homework/hw08}{这里} 。

对于直流特性的 \(I_C - V_{CE}\) 图像,如 \figref{03} 。

\qfig[03]{08p3.png}{\(I_C - V_{CE}\) 图像}

在设定不同的输入幅度时, 其有效值如\tabref{t1} 。在 \(15 V\) 输入时,出现了严重的失真,如 \figref{04} 。

\begin{table}[htp]
    \caption{不同输入电压对应的输出幅度和对应增益}
    \centering
    \label{t1}
    \begin{tabular}    {cccc}
    \toprule
    AMP & R1    & R2       & gain        \\
    \midrule
1   & 0.357 & 3.28E-02 & 91.8767507  \\
2.5 & 0.897 & 8.03E-02 & 89.5206243  \\
5   & 1.8   & 1.74E-01 & 96.66666667 \\
7.5 & 2.7   & 2.47E-01 & 91.48148148 \\
10  & 3.6   & 3.29E-01 & 91.38888889 \\
15  & 5.45  & 4.50E-01 & 82.56880734 \\
    \bottomrule
    \end{tabular}
\end{table}


\qfig[04]{08p4.jpg}{放大失真}

进一步的,考察其放大特性,如 \figref{05} 。找到下降到 \(\sqrt{2}/2\) 的时候的带宽,经过大量的采样,分布在 \(1.4 MHz\) 和 \(1.5 MHz\) 之间,和通常值偏差较小。由于 \(R_1\) 的电压在不同频率下保持几乎恒定,因此增益几乎是 \(V_{R2}/100 / (V_{R1,c}/100k) = V_{R2} \cdot 1k / 3.6\) ,如 \figref{06} ,那么特征频率为 \(1.4 M \cdot 84.7 = 118.58 MHz \) 。

\qfig[05]{08p5.png}{\(V_{R2}\)频率响应}

\qfig[06]{08p6.png}{\(Gain\)频率响应}


\section{实验结果分析}

\subsection{直流特性}

集电极电流随着 \(V_{CE}\) 的增长依次经过饱和区和放大器,这也解释了为什么在 \(0.1 - 1 V\) 区间内,电流的迅速增长;在之后电流基本稳定,是基极电流的稳定倍数的增大,而由于基极调制效应的存在,放大区曲线存在一定的上翘。

\subsection{交流特性}

交流特性是通过并联的回路实现的,通过并联提供小信号的电流源与电流检测,实现放大效果的检测。

在同频率不同幅度的基极输出电流下,放大倍数基本稳定。而在摆幅过大时,由于基极电流触底,产生失真,由于反向放大,因此显示为顶部失真。

\subsection{频率特性}

由于电容在高频的馈通效应,放大器的增益往往会随着频率上升而下降。这电容主要是两个 PN 结提供的。

\section{总结与思考}

\subsection{总结}

信号的放大电路是工作在静态工作点的,小信号需要依托在静态工作点上工作,为了提供小信号的附加输入,可以通过并联电流源或者串联电压源的方式提供。

\subsection{思考题}

Q1:频率特性测量时 \(E_C\) 应设置为多少伏?

为了使得信号波形工作在正常放大区, \(E_C\) 不能过小,如 \(0.5 V\) ;为了使得同样的小信号输出电流下, \(i_B\) 不致过大,不能选择较大的 \(E_C\) 如 \(20 V\) 。这就是为何选择 \(15 V\) 。

Q2:电阻Rc的直流分压如何随 \(E_C\) 变化?

在 \(E_C\) 较小,处于饱和区时,电流急剧上升,分压也随之上升; \(E_C\) 较大,进入放大区,则会因为基区调制,缓慢上升。

Q3: 当交流输入信号 \(V_B\) 过大时会出现什么

这对应着电流的摆幅也会过大,这样可能会造成电流的摆动使得管子不能工作在放大区而截止,产生输入电流的触底,由于反向放大,会产生顶部失真。

% Start Here

% End Here

\end{document}

% \begin{longtable}{ccc}
%     \caption{直流电路的 \(E_C - V_{CE} - I_C\) 特性}
%     \label{tab:my-tabletab:my-table}
%     % \begin{tabular}
%     \hline
%     EC  & VCE      & IC       \\
%     \hline
%     0   & 0        & 0        \\
%     0.1 & 0.018    & 7.36E-05 \\
%     0.2 & 0.026    & 1.57E-04 \\
%     0.3 & 0.033    & 2.42E-04 \\
%     0.4 & 0.038    & 3.27E-04 \\
%     0.5 & 0.044    & 4.13E-04 \\
%     0.6 & 0.048    & 4.98E-04 \\
%     0.7 & 0.053    & 5.95E-04 \\
%     0.8 & 0.057    & 6.73E-04 \\
%     0.9 & 0.062    & 7.60E-04 \\
%     1   & 0.065    & 8.47E-04 \\
%     2   & 0.103    & 1.87E-03 \\
%     3   & 0.139    & 2.83E-03 \\
%     4   & 0.195    & 3.77E-03 \\
%     5   & 0.388    & 4.59E-03 \\
%     6   & 0.759    & 5.21E-03 \\
%     7   & 1.494    & 5.47E-03 \\
%     8   & 2.426    & 5.53E-03 \\
%     9   & 3.364    & 5.59E-03 \\
%     10  & 4.308    & 5.65E-03 \\
%     11  & 5.25     & 5.71E-03 \\
%     12  & 6.2      & 5.76E-03 \\
%     13  & 7.14     & 5.81E-03 \\
%     14  & 8.09     & 5.86E-03 \\
%     15  & 9.05     & 5.92E-03 \\
%     16  & 9.99     & 5.97E-03 \\
%     17  & 1.08E+01 & 6.01E-03 \\
%     18  & 11.88    & 6.08E-03 \\
%     19  & 12.84    & 6.10E-03 \\
%     20  & 13.78    & 6.18E-03
%     \hline
%     % \end{tabular}
% \end{longtable}


% \begin{longtable}{lll} 
%     \caption{变换对} \\ 
%     \toprule
%     时域函数 & \(z\)域函数 & ROC \\
%     \midrule
%     \endfirsthead
    
%     \toprule
%     EC  & VCE      & IC       \\
%     \midrule
%     \endhead 
  
%     \hline
%     \multicolumn{3}{c}{见下页}\\   \bottomrule
%     \endfoot
  
%     \bottomrule
%     \endlastfoot

%     % \hline
%     0   & 0        & 0        \\
%     0.1 & 0.018    & 7.36E-05 \\
%     0.2 & 0.026    & 1.57E-04 \\
%     0.3 & 0.033    & 2.42E-04 \\
%     0.4 & 0.038    & 3.27E-04 \\
%     0.5 & 0.044    & 4.13E-04 \\
%     0.6 & 0.048    & 4.98E-04 \\
%     0.7 & 0.053    & 5.95E-04 \\
%     0.8 & 0.057    & 6.73E-04 \\
%     0.9 & 0.062    & 7.60E-04 \\
%     1   & 0.065    & 8.47E-04 \\
%     2   & 0.103    & 1.87E-03 \\
%     3   & 0.139    & 2.83E-03 \\
%     4   & 0.195    & 3.77E-03 \\
%     5   & 0.388    & 4.59E-03 \\
%     6   & 0.759    & 5.21E-03 \\
%     7   & 1.494    & 5.47E-03 \\
%     8   & 2.426    & 5.53E-03 \\
%     9   & 3.364    & 5.59E-03 \\
%     10  & 4.308    & 5.65E-03 \\
%     11  & 5.25     & 5.71E-03 \\
%     12  & 6.2      & 5.76E-03 \\
%     13  & 7.14     & 5.81E-03 \\
%     14  & 8.09     & 5.86E-03 \\
%     15  & 9.05     & 5.92E-03 \\
%     16  & 9.99     & 5.97E-03 \\
%     17  & 1.08E+01 & 6.01E-03 \\
%     18  & 11.88    & 6.08E-03 \\
%     19  & 12.84    & 6.10E-03 \\
%     20  & 13.78    & 6.18E-03
% \end{longtable}
