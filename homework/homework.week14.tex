\documentclass[lang=cn,11pt,a4paper,cite=authoryear]{elegantpaper}

\input{needed.tex}

\title{微电子器件实验\quad 基本运算电路}
\author{范云潜, 学号:18373486,搭档:徐靖涵,教师:彭守仲}
\institute{微电子学院 184111 班}
\date{\zhtoday}

\begin{document}

\maketitle

% \tableofcontents

\section{实验目的}

本组实验通过对运放的负反馈设计完成对信号的基本算数处理,如加减与积分等,加深对运放使用与信号负反馈的理解,掌握设计与使用能力。

\section{实验所用设备及器件}

主要设备有:电压源,任意波形发生器,示波器,台式万用表,相关线缆等,主要器件有四运放集成电路LM324N、电容、电阻。

\section{实验基本原理及步骤}

\subsection{加减运算电路}

对于如\figref{01}的加减运算电路来说,由于电阻与算数电路总的理想运放均为线性元件,因此其他数目的输出均可通过线性定理求得。此外,为了电路的差分输入的匹配,需要满足电阻关系:

\begin{equation}\label{eq1}
    R_1 // R_2 // R_f = R_3 // R_4 // R_5 
\end{equation}

\qfig[01]{1401.png}{加减运算电路}

首先求取负输入端的信号对输出的贡献,以 \(u_{I1}\) 为例,根据线性定理将其他输入置零,由于运放的虚断特性以及负端的基尔霍夫电流定律有:

\begin{equation}
    \begin{aligned}
        \frac{u_{I1}}{R_1} &= -\frac{u_O}{R_f} \\ 
        u_O &= -\frac{u_{I1}}{R_1} R_f
    \end{aligned}
\end{equation}
得到减法运算电路的关系。

接下来求取正输入端的信号对输出的贡献,以 \(u_{I3}\) 为例,根据线性定理将其他输入置零,由于运放的虚短特性以及负端的分压和虚断特性有:

\begin{equation}
    \begin{aligned}
        u_{-} = u_{+} &= u_{I3} \frac{R_4 // R_5}{R_3 + R_4 // R_5} = u_{I3} \frac{R_3 // R_4 // R_5}{R_3} \\
        u_{-} &= u_{O} \frac{R_f}{R_f + R_1 // R_2} = u_{o} \frac{R_f // R_1 // R_2}{R_1 // R_2}
    \end{aligned}
\end{equation}
联立\eqref{eq1},得到加法运算电路关系:

\begin{equation}
    u_O = \frac{u_{I3}}{R_3} R_f 
\end{equation}

综上,根据线性关系:

\begin{equation}
    u_{O}=R_{f}\left(\frac{u_{I 3}}{R_{3}}+\frac{u_{I 4}}{R_{4}}-\frac{u_{I 1}}{R_{1}}-\frac{u_{I 2}}{R_{2}}\right)
\end{equation}

任何条件下,存在加法电路关系,在匹配条件 \eqref{eq1} 满足时,存在减法电路关系。需要不同输入个数时,只需对端口进行扩展并满足匹配关系即可。

\subsection{积分运算电路}


对于如\figref{02}的积分运算电路来说,有如下的关系:

\begin{equation}
    \frac{u_I}{R} = \frac{\dd{Q}}{\dd{t}} = C \frac{\dd{u_O}}{\dd{t}}
\end{equation}

解以上的微分方程,在时间范围 \([t_1, t_2]\) 之间:

\begin{equation}
    u_O = \int_{t_1}^{t_2} u_I(t) \dd{t} + u_O(t_1)
\end{equation}

\qfig[02]{1402.png}{积分运算电路}

\subsection{实验步骤}

接下来,按照之前的理论推导进行各个电路的搭建,前四个电路由于 10k 电阻比较多,主要使用 10k 电阻,如\figref{03},\figref{04},\figref{05},\figref{06};后四个则使用 1k 电阻搭配 10k 电阻,如\figref{07},\figref{08},\figref{09},\figref{10}。


\qfig[03]{1403.png}{反向单倍放大电路}

\qfig[04]{1404.png}{反向双倍放大电路}

\qfig[05]{1405.png}{正向单倍放大电路}

\qfig[06]{1406.png}{正向双倍放大电路}

\qfig[07]{1407.png}{反向求和电路}

\qfig[08]{1408.png}{同向求和电路}

\qfig[09]{1409.png}{加减运算电路}

\qfig[10]{1410.png}{积分电路}

按照以上电路搭建实际电路,在 LM324N 的正负极分别接入 \(5 V\) 和 \(-5 V\) 电压,使用任意波形发生器进行输入信号产生即可,可以选用波形或者偏压两种方式。记录不同电路的输入输出数值。

\section{实验数据记录}

原始数据请见 \href{https://github.com/PannenetsF/Mirco-Electronic-Device-Experiment/tree/main/homework/hw14}{这里} 。

其中积分的波形如\figref{11},\figref{12},\figref{13}。

\qfig[11]{1411.jpg}{正弦波积分波形}

\qfig[12]{1412.jpg}{锯齿波积分波形}

\qfig[13]{1413.jpg}{方波积分波形}

\section{实验结果分析}

以上的实验结果基本表明,在电阻连接正确的情况下,本运放可以实现功能。

\section{总结与思考}


Q1. 在我们的实验中如何提高反相比例运算电路的放大倍数?

调大反馈电阻 \(R_f\) 或者调小负载电阻 \(R_i\) ,同时使得电阻匹配。

Q2. 如果放大倍数过高,会带来什么不利影响?

对噪声的放大可能会干扰正常波形。

Q3. 如何搭建微分电路?典型输入输出波形是什么样?

微分电路主要依赖于电容和电感,按照需要的方程将微分项替换为与 \(C\) 或者 \(L\) 相关的方程即可。典型的波形是积分波形,输入方波出现三角波,输出三角波出现二次函数波等。

% Start Here

% End Here

\end{document}